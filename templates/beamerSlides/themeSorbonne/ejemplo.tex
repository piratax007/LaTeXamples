%%--------------------------------------------------------------------
%% MSc. Fausto M. Lagos S.
%% @piratax007 - 2020 - GPLv3+
%% Presentación de ejemplo de uso de la plantilla beamerthemesorbonne
%%--------------------------------------------------------------------

\documentclass[compress,xcolor=table]{beamer}

%% Paquetes adicionales
\usepackage[spanish]{babel}
\usepackage[utf8]{inputenc}
\usepackage[T1]{fontenc}
\usepackage{datetime}
\usepackage{amsthm}
\usepackage{amsmath}
\usepackage{amssymb}
\usepackage{amsfonts}
\usepackage{subcaption}
\usepackage{tikz}
\usetikzlibrary{calc, arrows, babel}

%% Opciones para el comando \usetheme
%%  - nosectionpages: Sin página entre secciones
%%  - flama: Utilizar la fuente flama, requiere xelatex/lualatex + la fuente para poder compilar
%%  - compressminiframes: Lleva al encabezado las marcas de página
\usetheme[compressminiframes]{sorbonne}

%% Página de Título
\title{Título de la presentación}
\foottitle{Título en el pie de la diapositiva} % Opcional, se imprime al final de las diapositivas, por defecto toma el mismo título.
\subtitle{Subtítulo de la presentación} % Opcional
\date{\today}
\author{Autor} % puede utilizar \\ para agregar otros detalles del autor
\institute{Filiación} % Opcional

% Biblatex (opcional)
\setbeamertemplate{bibliography item}{\rmfamily\insertbiblabel}
\usepackage[backend=bibtex, style=numeric, citestyle=numeric]{biblatex}
\bibliography{library.bib}
\renewcommand*{\bibfont}{\footnotesize}


%%-------------------------------------------
%% DIAPOSITIVAS
%%-------------------------------------------

\begin{document}

\begin{frame}[plain]
	\titlepage
	\setcounter{framenumber}{0}
\end{frame}

\section{Introducción} \subsection{}

\begin{frame}{Aquí va el título de la diapositiva} % puede utilizar \insertsection o \insertsubsection para incluir el mísmo título de sección o subsección
	Esta presentación utiliza la plantilla \textbf{beamerthemesorbonne.sty}. Para cambiar el \textit{background} de la página de título reemplace el archivo \emph{background.pdf} en el directorio \emph{images}.
	
	\begin{alertblock}{Directorio images}
		Todas las imágenes que vaya a utilizar en la presentación póngalas en el directorio \textbf{images}.
	\end{alertblock}
	
	Para modificar el \textbf{logo} que aparece en la esquina superior derecha de cada diapositiva, ponga la imágen en el directorio \emph{images} y modifique la línea 273 en la sección \textit{Frametitle} del archivo \emph{beamerthemesorbonne.sty}.
	
	Para modificar los colores altere los valores de los colores en la sección \emph{Color} del archivo \emph{beamerthemesorbone.sty}. Si altera los nombres tendrá que modificar también los nombres en la lista de comandos \emph{setbeamercolor}.
\end{frame} 

\section{Utilice columnas} \subsection{}

\begin{frame}{Título de la diapositiva}
	\begin{columns}
  		\begin{column}{.45\textwidth}
  			Contenido de la primera columna
  		\end{column}
  		\begin{column}{.45\textwidth}
  			Contenido de la segunda columna
  		\end{column}
  \end{columns}
\end{frame}

\section{Utilice bloques}

\begin{frame}[fragile]{Bloques}
	
	\begin{block}{Título del bloque}
		Contenido del bloque
	\end{block}
	
	\begin{exampleblock}{Título del bloque} 
		Contenido del bloque
	\end{exampleblock}
	
	\begin{alertblock}{Título del bloque}
		Contenidod el bloque
	\end{alertblock}
\end{frame}

\begin{frame}{citas bibliográficas}
	Esta diapositiva muestra el ejemplo de uso de citas \cite{Leithold}
\end{frame}

\section{Referencias}

\begin{frame}[allowframebreaks]{Referencias}
	\printbibliography%[heading=none]
\end{frame}

\end{document}

